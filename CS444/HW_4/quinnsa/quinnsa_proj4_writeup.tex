\documentclass[letterpaper,10pt,notitlepage,fleqn]{article}

\usepackage{nopageno} %gets rid of page numbers
\usepackage{alltt}                                           
\usepackage{float}
\usepackage{color}
\usepackage{url}
\usepackage{balance}
\usepackage[TABBOTCAP, tight]{subfigure}
\usepackage{enumitem}
\usepackage{pstricks, pst-node}
\usepackage{geometry}
\geometry{textheight=9in, textwidth=6.5in} %sets 1" margins 
\newcommand{\cred}[1]{{\color{red}#1}} %command to change font to red
\newcommand{\cblue}[1]{{\color{blue}#1}} % ...blue
\usepackage{hyperref}
\usepackage{textcomp}
\usepackage{listings}

\def\name{Sam Quinn}

\parindent = 0.0 in
\parskip = 0.2 in

\title{Project 4 Write Up}
\author{Sam Quinn}

\begin{document}
\maketitle
\hrule

\section*{What do you think the main point of this assignment is?}
    I think that the main purpose of this assignment was to inspect and understand 
    how the ``Simple list of Blocks'' allocator works. Once we have a clear 
    understanding of how one of the more simple memory allocators in th Linux kernel 
    work we are given the chance to play with it. 
\section*{How did you personally approach the problem? Design decisions, algorithm, etc.}
    I personally first began by researching how the \textbf{slob} allocator works. In the 
    \textit{slob.c} file at the top there was a very good description of how the 
    \textbf{slob} allocator works. The original \textbf{slob} allocator works on a first fit algorithm 
    that will allocate the first space it comes across that is large enough to fit 
    the current request. To implement a ``Best fit algorithm'' not much had to be 
    changed. All we did was instead of attempting to allocate the first space that 
    fits our algorithm will scan all of the available blocks on the current page 
    and store the location of the one that most closely matches that of the request 
    size. \\
    We at first did not ensure that our algorithm verified that it found the 
    best page of blocks. To remedy this mistake we added another function that does 
    essentially the same thing as the \textit{slob\_page\_alloc()} function but 
    only returns the size of the block minus the size of the request. Our implementation's 
    performance was greatly impacted since it needs to scan all of the blocks in 
    each page then scan all of the blocks again. The only way our algorithm's speed 
    is increased is that if it finds an exact fit it will short circuit and attempt 
    to allocate that space immediately.

\section*{How did you ensure your solution was correct? Testing details, for instance.}
    To ensure that our algorithm was indeed finding the best possible fit we used 
    printk messages. We first print the size of the request, then we print each 
    available space in the best page. Once either all of the blocks were examined 
    or if an exact fit was found we print the ``Best Fit''. It is pretty easy to 
    see with all the sizes printed out which is the best fit.
\section*{What did you learn?}
    I learned how a memory allocator works in Linux. After a few times compiling 
    with printk statements every request I quickly learned that there are a lot of 
    memory allocation request going on! I eventually changed our code to only print 
    every 10,000 requests which is still quite often. 
    
\end{document}
