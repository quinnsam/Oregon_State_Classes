\documentclass[letterpaper,10pt,notitlepage,fleqn]{article}

\usepackage{nopageno} %gets rid of page numbers
\usepackage{alltt}                                           
\usepackage{float}
\usepackage{color}
\usepackage{url}
\usepackage{balance}
\usepackage[TABBOTCAP, tight]{subfigure}
\usepackage{enumitem}
\usepackage{pstricks, pst-node}
\usepackage{geometry}
\geometry{textheight=9in, textwidth=6.5in} %sets 1" margins 
\newcommand{\cred}[1]{{\color{red}#1}} %command to change font to red
\newcommand{\cblue}[1]{{\color{blue}#1}} % ...blue
\usepackage{hyperref}
\usepackage{textcomp}
\usepackage{listings}
\usepackage{graphicx}
\usepackage{tikz}
\usetikzlibrary{shapes,arrows}
\usepackage{cite}

\def\name{Sam Quinn}

\parindent = 0.0 in
\parskip = 0.2 in

\title{Annotated Bibliography}
\author{Sam Quinn}

\begin{document}
\maketitle
\hrule

\section{Processes and Threads}
\textbf{Linux} \\ \\ \\
\textit{Linux Kernel Development, 3rd Edition} is apart 
of the \textit{Developer's Library} which are a set of books that are written by 
expert technology practitioners who are especially skilled at organizing and 
presenting information in a way that's useful for other programmers. While writing 
the ``Process and Threads'' section of my report I looked at Chapter 3 ``Process 
Management''. Chapter 3 covers the life of processes and threads within a Linux system 
and how they function on a lower level than the abstraction we are familiar with. 
Reading this chapter has given me the information needed to compare how a process 
and thread differers from Linux and Windows ~\cite{LKD3}.
\\ \\
\textit{Linux Man Pages} is a collection of manuals created by developers to explain Linux functions.
Every manual is installed with the program or are preinstalled for Linux system calls. 
These manual pages are a great way to get information 
about Linux commands and are very easy to access from any Linux machine. I specifically 
used the \textit{Linux Man Pages} for the system calls used by Linux to create both process and threads. 
My \textit{Linux Man Page} research consisted of reading the manuals for \textbf{fork()} and
\textbf{clone()}. The information gathered gave more detail on the specific flags 
\textbf{clone()} needs to create either a thread or a process ~\cite{LMP}.

\textbf{Windows} \\ \\ \\
\textit{Windows Internals, 5th Edition} is intended for advanced computer professionals 
who want to understand how the core components of the Windows operating system 
works internally. This book has three Microsoft developers that have incorporated 
their knowledge into this book which increase credibility and accuracy. While 
comparing how processes and threads are created in both Linux and Windows I specifically 
examined \textit{Windows Internals} Chapter 5. 
Chapter 5 ``Processes, Threads, and Jobs'', which covers all of the relevant information 
about the creation, data structures used, and termination of both processes and 
threads within a windows system ~\cite{WI5}. 
\\ \\
\textit{Microsoft TechNet} is hosted on the Microsoft.com domain which means that 
it has been approved by the Microsoft team as an adequate source of information. The 
TechNet Library contains technical documentation for IT professionals using Microsoft 
products , tools, and technologies. For the ``Process and Threads'' section of my 
report I specifically referenced Chapter 3 ``Developing Phase: Process and Thread 
Management'' from the \textit{UNIX Custom Application Migration Guide} document. 
This chapter went into details about the fundamental differences in process and 
thread implementations of UNIX and Windows applications ~\cite{TN}. 

\section{File Systems}
\textbf{Linux} \\ \\ \\
\textit{Linux Kernel Development, 3rd Edition} Chapter 13 ``The Virtual Filesystem''
explains the process of how Linux interacts with filesystems. This chapter covers 
in great detail how Linux takes advantage of a \textbf{VFS} (Virtual Filesystem) 
that abstracts the underlying filesystem completely. From this chapter I acquired 
information that I can use to contrast the filesystem models of Windows and Linux. 
The \textbf{VFS} is one of the more unique functions of Linux which allows Linux 
to support many different filesystem structures 
~\cite{LKD3}. 


\textbf{Windows} \\ \\ \\
\textit{Windows Internals, Part 2, 6th Edition} Chapter 12 ``File Systems'' explains 
in detail how Microsoft Windows filesystems are implemented and the differences 
between the common filesystems. The section of this chapter I was most interested 
in was about the ``File System Diver Architecture'' which explains how the operating 
system interacts with the filesystem through the kernel. There are two different 
filesystem drivers that I will be comparing, local and remote.~\cite{WI26}.

\textit{Windows Internals, Part 2, 6th Edition} Chapter 8 ``I/O System'' gives some 
more specific detail  about how Windows preforms certain I/O operations to a 
filesystem. The majority of the information I used from this chapter during the 
``Filesystems'' section was how Windows handles Diver objects and Device objects, which, 
are a critical portion of the functionality of a Windows filesystem. From this chapter
I was able to compare the number of drivers needed between Windows and Linux 
~\cite{WI26}.



\section{CPU Scheduling}

\textbf{Linux} \\ \\ \\
\textit{Linux Kernel Development, 3rd Edition} Chapter 4 ``Process Scheduling'' 
discusses how the Linux CPU is scheduled and the various scheduling algorithms used.  
This information was useful in my research for the Linux portion of the CPU scheduling 
comparison. With \textit{Linux Kernel Development, 3rd Edition} and the infromation I found
about how the CPU is 
scheduled for multitasking, the scheduling algorithms used, how context switching is executed, and scheduling  
related system calls function I was able to find similarities and differences between Linux and Windows. This chapter also displays many of the data structures and 
functions that the kernel uses to manipulate the current CPU scheduler ~\cite{LKD3}. 

\textbf{Windows} \\ \\ \\
\textit{Windows Internals, Part 1, 6th Edition} Chapter 5 ``Processes, Threads, and Jobs'' 
covers a multitude of topics about processes, threads, and jobs with very detailed 
descriptions of how the kernel handles each of these individually. The section I 
mainly focused on for the purpose of CPU Scheduling was the ``Overview of Windows 
scheduling''. This section covers the topics priority levels, context switching, 
and scheduling scenarios that I would like to compare to the Linux implementation. 
Because Windows implements their own CPU scheduler there is no information covering 
the benefits and impairments of different algorithms ~\cite{WI16}. 
\\


\bibliography{ref}{}
\bibliographystyle{plain}


\end{document}
