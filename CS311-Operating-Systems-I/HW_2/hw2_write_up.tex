\documentclass[letterpaper,10pt,notitlepage,fleqn]{article}

\usepackage{nopageno} %gets rid of page numbers
\usepackage{alltt}                                           
\usepackage{float}
\usepackage{color}
\usepackage{url}
\usepackage{balance}
\usepackage[TABBOTCAP, tight]{subfigure}
\usepackage{enumitem}
\usepackage{pstricks, pst-node}
\usepackage{geometry}
\geometry{textheight=9in, textwidth=6.5in} %sets 1" margins 
\newcommand{\cred}[1]{{\color{red}#1}} %command to change font to red
\newcommand{\cblue}[1]{{\color{blue}#1}} % ...blue
\usepackage{hyperref}
\usepackage{textcomp}
\usepackage{listings}

\def\name{Sam Quinn}

\parindent = 0.0 in
\parskip = 0.2 in

\title{Project 2 Write Up}
\author{Sam Quinn}

\begin{document}
\maketitle
\hrule

\section*{Design}

\subsection*{Body of Program}
Collect command line args using getopts\\
Project note: it is a requirement to handle multiple file names as arguments.

\begin{itemize} 
\item if no options/arguments, print usage
\item if some option + v, use verbose output
\item create array of option arguments and pass into calls to specific functions for each option (?)
\item also pass in variable set 1 or 0 for verbose output
\end{itemize}

\subsection*{Functions of Program}

\subsubsection*{-t Option}
print a concise table of contents of the archive.\\
Project notes: error if the archive does not exist.


\begin{itemize}
\item open archive file, if it doesn't exist, error
\item read archive printing the names of each file found
\item if verbose is enabled then print permissions, user/group, size, date, and file name 
\end{itemize}

\subsubsection*{-q     Option}       
-q quickly append named files to archive\\
Project notes: For the -q command myar should create an archive file if it doesn't exist, using permissions ``666''. For the other commands myar reports an error if the archive does not exist, or is in the wrong format.
\begin{itemize}
\item if the archive has not been created then create it
\item if the archive is present then open the file in append mode
\item copy the contents of the file in to the archive
\item repeat for all files listed in the command line
\end{itemize}

\subsubsection*{-x     Option}
-x extract named files\\
Project notes: error if the archive does not exist, only extract the first listing
    
\begin{itemize}
\item open the archive file, error if it doesn't exist
\item read over the ARMAG string
\item compare the name of file you wish to extract with file name in the archive
\item if file in the archive matches the file you wish to extract copy the contents to an output file with the same name as it is in the archive
\item if the file in the archive doesn't match it checks the next by adding the size to the offset using lseek
\item if the file is not present in the archive exit(-1) if verbose is enabled prompt that no member "file" was found
\item repeat for all files listed in the command line
\end{itemize}

\subsubsection*{-d     Option}    
-d	delete named files from archive\\
Project notes: error if the archive does not exist, only delete the first occurrence of the name.

\begin{itemize}
\item open the archive file, error if it doesn't exist
\item scan file names till it finds the one you are trying to delete
\item find the file info you are trying to delete e.g.(size, name)
\item save the archive - the file you are trying to delete to buffer
\item open output file (the original archive)
\item write buffer to it with truncate enabled
\item if the file name doesn't match any files in the archive exit(-1)
\item if verbose is enabled and the file wasn't found prompt that there was no member named "file"
\item repeat for all files listed in the command line
\end{itemize}

    
\subsubsection*{-A     Option}
-A quickly append all ``regular'' files in the current directory (except the archive itself)\\
Project notes: error if the archive does not exist, should not include the archive or the executable.

\begin{itemize}
\item open the archive file, error if it doesn't exist
\item use readdir() to read the '.'(current directory)
\item use the struct given by readdir() to find all regular files names
\item exclude the archive itself and the executable
\item call q for every regular file in the current directory

\end{itemize}

\section*{Design Deviations}
I didn't deviate from the programs design described above at all. My program runs almost identical to the ar function. 

\section*{Work Log}
0fea957,Sam Quinn,78 minutes ago,Final Product perfect!\\
-Finished the program completly with no errors and everything works.\\
21528ec,Sam Quinn,2 hours ago,ALL DONE Need touch ups.\\
-Finished but need to get rid of my little debug features.\\
834a906,Sam Quinn,23 hours ago,Works what's Up'\\
-Delete finally works .\\
f6d3ac3,Sam Quinn,24 hours ago,Almost done delete still doesnt work right.\\
-All other functionalitly works but having problems with the delete function.\\
8e7eb4f,Sam Quinn,4 days ago,Weird problems happing\\
-Weird errors and incorrect outputs from trivial things. (Before I broke up the display function)\\
e8eaeb6,Sam Quinn,6 days ago,Testing GIT a little bit.\\
-Not much more progress from the privious just playing around with GIT a bit.\\
ff0eaf8,Sam Quinn,6 days ago,Almost got -vt\\
-Almost completed with the verbose function or -t.\\
4a54f71,Sam Quinn,6 days ago,Initial commit\\
-Have functionality of -q but still is my first commit.


\section*{Challenges Overcame}
This project brought on a lot of challenges for me all of which I had overcame. One challenge was time management I made sure I worked on this assignment a little bit every day. I also gave myself deadlines to complete certain  requirements, for example i would tell myself that i had to complete the extract function by tomorrow. This helped keep me motivated and on track to completing this project in time. I had also overcome many small challenges within the code itself. The two most notable was figuring out how to print the date in the proper way and getting my delete function to work the way it should. I solved the date printing problem by reading the textbook (which wasn't my first resource, which now it is) and i solved the delete function problems by drawing out what needs to be done and figuring the math out on paper rather than in my head.
\section*{Questions}

\textbf{What do you think the main point of this assignment is?}\\
I think the main point of this assignment was to work with system calls, become very familiar with file I/O, and most importantly to challenge us to see what we can do.  
    
\textbf{How did you ensure your solution was correct?}\\
The most important way I made sure my solution was the correct way was I checked my archive with the Unix ar command to make sure the archive was formatted correctly. When the ar command would return "archive malformed" and I couldn't visually inspect the differences between the archive created by my function and the Unix ar command I took advantage of the diff command to show where the differences are. 
    
\textbf{What did you learn?}\\
I learned how to use the getopt function to handle my command line arguments. I also learned how cool file descriptors are and how they work and what functions they work with. I touched up on my c programming skills. I learned that I can program a fairly intense program which is good to know I can do it. I learned a lot about the information of files e.g the stat info and what permissions let you do what. I really enjoyed this project I feel very accomplished knowing that i have finished it with all of the functionality complete.
    
\end{document}