\documentclass[letterpaper,12pt,notitlepage,fleqn]{article}

\usepackage{nopageno} %gets rid of page numbers
\usepackage{alltt}                                           
\usepackage{float}
\usepackage{color}
\usepackage{url}
\usepackage{balance}
\usepackage[TABBOTCAP, tight]{subfigure}
\usepackage{enumitem}
\usepackage{pstricks, pst-node}
\usepackage{geometry}
\geometry{textheight=9in, textwidth=6.5in} %sets 1" margins 
\newcommand{\cred}[1]{{\color{red}#1}} %command to change font to red
\newcommand{\cblue}[1]{{\color{blue}#1}} % ...blue
\usepackage{hyperref}
\usepackage{textcomp}
\usepackage{listings}
\usepackage{amsmath}
\usepackage[usestackEOL]{stackengine}
\usepackage{multicol}

\def\name{Sam Quinn}

\parindent = 0.0 in
\parskip = 0.2 in

\title{Homework \#2}
\author{Sam Quinn}

\begin{document}
\maketitle
\hrule

\section{}
\textbf{Which of the following are negligible functions? Justify your answers.}
\begin{itemize}
	\item $\sqrt{\frac{\lambda}{2^{\lambda}}} \rightarrow 0$ - Is \textbf{negligible}, the denominator grows exponentially while the numerator grows linear making this equation approach $0$.
	\item $\frac{1}{2^{log(\lambda^{2})}} \rightarrow \infty$ - Is \textbf{Not negligible}, as the values of $\lambda$ continue to increase there are polynomial values that could grow faster in the numerator. Thus this equation approaches $\infty$.
	\item $\frac{1}{\lambda^{log(\lambda)}} \rightarrow 0$ - Is \textbf{negligible}, the denominator will grow faster than the numerator. As $\lambda$ gets larger it is also raised to the $log$ of itself which will eventually on the path to $\infty$ surpass any numerator making it approach $0$.
	\item $\frac{1}{\lambda^{2}} \rightarrow \infty$ - Is \textbf{not negligible}, because the denominator in this equation grows linearly there are numerators raised to a constant that could grow faster than the denominator. This equation approaches $0$.
	\item $\frac{1}{2^{(log\lambda)^{2}}} \rightarrow 0$ - \textbf{Is negligible}, because the denominator will grow exponentially while the numerator will grow linear. This equation will approach $0$.
	\item $\frac{1}{\sqrt{\lambda}} \rightarrow \infty$ - This is \textbf{not negligible}, the numerator will grow much faster than the denominator which is a $\sqrt{\lambda}$. This equation approaches $0$.
	\item $\frac{1}{2^{\sqrt{\lambda}}} \rightarrow 0$ - Is \textbf{negligible}, even though the denominator has an exponent that is a $\sqrt{\lambda}$ it is still a positive exponent making the denominator grow faster than the numerator. This equation approaches $0$.
\end{itemize}
\section{}
\textbf{Let $G:\lbrace0,1\rbrace^{\lambda} \rightarrow \lbrace0,1\rbrace^{2\lambda}$ be a length doubling PRG, and consider then algorithm $H:\lbrace0,1\rbrace^{\lambda} \rightarrow \lbrace0,1\rbrace^{3\lambda}$ given below:}
\begin{center}
\begin{multicols}{2}
\framebox{
\Longstack[l]{
	$\iota_{PRG-real}$ \\
	$\textbf{\underline{H(s):}}$ \\
	\hspace{15pt}$x:=G(s)$ \\
	\hspace{15pt}$y:=G(0^{\lambda})$ \\
	\hspace{15pt}return $x \bigoplus y $\\
	}
}
\\ $\Downarrow$
\columnbreak
\begin{flushright}


\framebox{	
\Longstack[l]{
	$\iota_{PRG-rand}$ \\
	$\textbf{\underline{H(s):}}$ \\
	\hspace{15pt}$x \leftarrow \lbrace 0,1 \rbrace^{2\lambda}$ \\
	\hspace{15pt}return $x$\\
}
}
\end{flushright}
\end{multicols}

\begin{multicols}{2}
\framebox{
\Longstack[l]{
	$\iota_{PRG-real}$ \\
	$\textbf{\underline{H(s):}}$ \\
	\hspace{15pt}$x:=A(s)$ \\
	\hspace{15pt}$y:=G(0^{\lambda})$ \\
	\hspace{15pt}return $x \bigoplus y $\\
	}
}
\columnbreak
\begin{flushright}


\framebox{	
\Longstack[l]{
	$\textbf{\underline{A(s):}}$ \\
	\hspace{15pt}$x \leftarrow \lbrace 0,1 \rbrace^{2\lambda}$ \\
	\hspace{15pt}return $x$\\
}
}
\end{flushright}
\end{multicols}
\begin{multicols}{3}
\framebox{
\Longstack[l]{
	$\iota_{PRG-real}$ \\
	$\textbf{\underline{H(s):}}$ \\
	\hspace{15pt}$x:=\lbrace 0,1 \rbrace ^{2\lambda}$ \\
	\hspace{15pt}$y:=G(0^{\lambda})$ \\
	\hspace{15pt}return $x \bigoplus y $\\
	}
}
\columnbreak
\Longstack[c]{
Based on OTP secrecy \\
$\approx$
}
\columnbreak
\framebox{	
\Longstack[l]{
	$\iota_{PRG-rand}$ \\
	$\textbf{\underline{H(s):}}$ \\
	\hspace{15pt}$x \leftarrow \lbrace 0,1 \rbrace^{2\lambda}$ \\
	\hspace{15pt}return $x$\\
}
}

\end{multicols}
\end{center}
In the first step we are able to detach the $x$ assignment in to a separate function named $A()$. Because we assume that $G()$ is a secure PRG that returns a uniformly random number every time we could make $A()$ a substitute with this assumption we can eliminate $s$. We \textbf{cannot} get $y$s value from $A()$ because if we pass the same value in each time the PRG will return the same random number. Because OTP security we can say that $H()$ is secure based on the fact that we always will have a random $x$ and when $\bigoplus$ with the same number $y$ it will provide total secrecy.
\section{}
\textbf{Show that $H$ is \textbf{not} a secure PRG (even if \textit{G} is). Describe a successful distinguisher for $\iota^{H}_{prg-real}$ and $\iota^{H}_{prg-rand}$. Explicitly compute its advantage.}
\begin{center}
\begin{multicols}{2}
\framebox{
\Longstack[l]{
$\iota_{PRG-real}$ \\
	$\textbf{\underline{H(s):}}$ \\
	\hspace{15pt}$x:=G(s)$ \\
	\hspace{15pt}return $s \Vert x $\\
}
}
\columnbreak
\begin{flushright}

\framebox{
\Longstack[l]{
$\iota_{PRG-rand}$ \\
	$\textbf{\underline{H(s):}}$ \\
	\hspace{15pt}$x \leftarrow \lbrace 0,1 \rbrace ^{3\lambda}$ \\
	\hspace{15pt}return $x$\\
}
}
\end{flushright}
\end{multicols}

\framebox{
\Longstack[l]{
	$\textbf{\underline{Query(s):}}$ \\
	\hspace{15pt}$x = $ First $\lambda$ bits of $s$ \\
	\hspace{15pt}$x' = $ Last $2\lambda$ bits of $s$ \\
	\hspace{15pt}$z = G(x)$ \\
	\hspace{15pt}$y = $ Last $2\lambda$ bits of output from $G(s)$ \\
	\hspace{15pt}return $x' == y$\\
}
}

\end{center}
To determine which world we are in we could create a distinguisher function named $Query()$ above. The first thing we would have to do is get the full output from $H()$ and pass that into $Query()$. Inside of $Query()$ we extract the first $\lambda$ bits off the front of $s$ and assign it to $x$ with the remaining going into $x'$. $Query()$ would pass $x$ into the same $G()$. If the output from $G()$ is the same as $x'$ we have determined that we are in the $\iota_{PRG-real}$ world.
\section{}
\textbf{Suppose $F$ is a secure PRF. Define the following function $F'$ as:\\ \\ $F'(k,x \Vert x')=F(k,x) \vert F(k,x \bigoplus x')$}\\\\
\textbf{$x$ and $x'$ are each in bits long, where in is the input length of $F$ . Show that $F'$ is \textbf{not} a secure PRF (even if $F$ is). Describe a distinguisher and compute its advantage.\\ \\
\textit{Hint:} Remember, you are not attacking $F$ . In fact, $F$ may be the best PRF in the world. You
are attacking the faulty way in which $F'$ uses $F$.}
\begin{multicols}{2}
\begin{center}
$x = 1111$ \\
$x' = 0000$ \\
$x \Vert x' = 11110000$
\end{center}
\columnbreak
$F'(k,x \Vert x')=F(k,x) \Vert F(k,x \bigoplus x')$ \\
$F'(k,x \Vert x')=1111 \Vert 1111 \bigoplus 0000$\\
$F'(k,x \Vert x')=1111 1111$
\end{multicols}
We can exploit $F'$ by taking advantage of the $\bigoplus$ with a zero string. Every string $\bigoplus$ with 0 returns itself. So now if we pass in a string where the second half of the string is all $0$'s then we can prove that we will get repeating results back consistently in this case all 1's.
	
\end{document}
