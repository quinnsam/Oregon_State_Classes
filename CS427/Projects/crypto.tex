\documentclass[letterpaper,11pt,notitlepage,fleqn]{article}

%\usepackage{nopageno} %gets rid of page numbers
\usepackage{alltt}                                           
\usepackage{float}
\usepackage{color}
\usepackage{indentfirst}
\usepackage{url}
\usepackage{balance}
\usepackage[TABBOTCAP, tight]{subfigure}
\usepackage{enumitem}
\usepackage{pstricks, pst-node}
\usepackage{geometry}
\geometry{textheight=9in, textwidth=6.5in} %sets 1" margins 
\newcommand{\cred}[1]{{\color{red}#1}} %command to change font to red
\newcommand{\cblue}[1]{{\color{blue}#1}} % ...blue
\usepackage{hyperref}
\usepackage{textcomp}
\usepackage{listings}
\usepackage{graphicx}
\usepackage{amsfonts}
\usepackage{amsmath}

% Code snippets color
\definecolor{dkgreen}{rgb}{0,0.6,0}
\definecolor{gray}{rgb}{0.5,0.5,0.5}
\definecolor{mauve}{rgb}{0.58,0,0.82}
\lstset{frame=tb,
  language=C,
  aboveskip=3mm,
  belowskip=3mm,
  showstringspaces=false,
  columns=flexible,
  basicstyle={\small\ttfamily},
  numbers=none,
  numberstyle=\tiny\color{gray},
  keywordstyle=\color{blue},
  commentstyle=\color{dkgreen},
  stringstyle=\color{mauve},
  breaklines=true,
  breakatwhitespace=true,
  tabsize=3
}
\lstdefinelanguage{diff}{
  morecomment=[f][\color{blue}]{@@},     % group identifier
  morecomment=[f][\color{red}]-,         % deleted lines 
  morecomment=[f][\color{green}]+,       % added lines
  morecomment=[f][\color{magenta}]{---}, % Diff header lines (must appear after +,-)
  morecomment=[f][\color{magenta}]{+++},
}
% End color
\def\name{Sam Quinn}

\parindent = 0.4444 in
\parskip = 0.2 in

\begin{document}
\begin{titlepage}
\vspace*{\fill}

\newcommand{\HRule}{\rule{\linewidth}{0.5mm}} % Defines a new command for the horizontal lines, change thickness here

\center % Center everything on the page

%----------------------------------------------------------------------------------------
%TITLE SECTION
%----------------------------------------------------------------------------------------

%\includegraphics[scale=.5]{image.eps}
\HRule \\[0.4cm]
{ \huge \bfseries CS427 Project}\\[0.4cm] % Title of your document

%----------------------------------------------------------------------------------------
%HEADING SECTIONS
%----------------------------------------------------------------------------------------

\textsc{\LARGE Socat Cryptography Analysis}\\[0.5cm] % Name of your university/college
\textsc{\Large CS427 Project}\\[0.5cm] % Major heading such as course name
\textsc{\large Winter 2016}\\[0.5cm] % Minor heading such as course title


\HRule \\[1.5cm]
%----------------------------------------------------------------------------------------
%AUTHOR SECTION
%------------------------------------ ----------------------------------------------------

\begin{minipage}{0.4\textwidth}
\begin{flushleft} \large
\emph{Student:}\\
        \textbf{Sam \textsc{Quinn}} \\ % Your name
        {\small Quinnsa@Oregonstate.edu}
        \end{flushleft}
        \end{minipage}
        ~
        \begin{minipage}{0.4\textwidth}
        \begin{flushright} \large
        \emph{Professor:} \\
            \textbf{Michael \textsc{Rosulek}} \\ % Supervisor's Name
            {\small Rosulekm@eecs.Oregonstate.edu}
            \end{flushright}
            \end{minipage}\\[3cm]

                %----------------------------------------------------------------------------------------
                %DATE SECTION
                %-----------------    -----------------------------------------------------------------------

{\large \today}\\[3cm] % Date, change the \today to a set date if you want to be precise

%----------------------------------------------------------------------------------------
%LOGO SECTION
%------   ----------------------------------------------------------------------------------

%\includegraphics{Logo}\\[1cm] % Include a department/university logo - this will require the graphicx package

%----------------------------------------------------------------------------------------

\vfill % Fill the rest of the page with whitespace



\end{titlepage}

\tableofcontents
\newpage

\section{Introduction}
\indent One common problem with networked computers is how files are transferred between them in a secure way. There are a few solutions out there, many secure and there are many that are not. In this analysis I am going to talk about how one particular mistake in Socat, a common Unix based network transfer application, compromised their entire security.
\\
\indent Socat is an application that can concatenate program output or files to a network socket. Socat can relay data in a bidirectional manner between two independent data channels. Socat can transmit to a file, pipe, device, or as socket (IP4, IP6, UDP, TCP) all encrypted over SSL. It is easy to see the need for SSL here since anytime you send data over a network any eavesdropper can view your data in plaintext, unless it is encrypted. 
\\
\\

\section{Problem Defined}
\indent Socat secures network transmission with the extremely popular OpenSSL. OpenSSL is in fact still considered very secure, Socats problems were the products of their own misuse of the secure libraries within OpenSSL. Socat has implemented a hybrid encryption scheme to securely transfer data among peers. The symmetric keys are obtained using Diffie Hellman key agreement (DHKA).
\\
\indent Diffie Hellman protocol needs to publicly share two variables $g$ and $p$ to function correctly. The variable $p$ is public and is very large since the $p$ value will be used as the modulus for the combined shared key after the two parties raise the $g$ to their private keys. When generating $p$ and $g$ you want to choose ``good numbers''. For example $p$ that is greater than $2048bits$ in length and that satisfy $p = 2t+1$ where $t$ is also prime are known to be more difficult to break.
When choosing $p$ values the most important thing is that both $p$ and $g$ must be prime, which is where Socat went wrong. Socat for a full year had a $p$ value that was not prime.
\\
\\
The Diffie Hellman p that was used for almost a year was:
\begin{lstlisting}
static unsigned char dh1024_p[] = {
	143319364394905942617148968085785991039146683740268996579566
	827015580969124702493833109074343879894586653465192222251909
	074832038151585448034731101690454685781999248641772509287801
	359980318348021809541131200479989220793925941518568143721972
	993251823166164933334796625008174851430377966394594186901123
	322297453
 }
\end{lstlisting}
\indent To the developers credit, the original number is not exactly easy to tell determine if it is prime or not. But given how easy it is to verify prime numbers, it should have been checked. OpenSSL has even made the process of generating Diffie Hellman primes easier by including the “dhparam” function which given the ‘-C’ flag will actually output c code that the developer could have just copy and pasted into Socat.
\\
\indent It was odd how this non-prime number got here in the first place. How could the developer at Socat mess this up unless he was trying? The development of Socat is open source so I went back to the exact commit message when this non-prime value was added.
\\
\begin{lstlisting}[language=diff]
--- a/CHANGES
+++ b/CHANGES
@@ -72,6 +72,9 @@ corrections:
 
        fixed a few minor bugs with OpenSSL in configure and with messages
 
+       Socat did not work in FIPS mode because 1024 instead of 512 bit DH prime
+       is required. Thanks to Zhigang Wang for reporting and sending a patch.
+
\end{lstlisting}
The commit where the non-prime was added was to replace the first $512 bit$ prime number due to an incompatibility with FIPS (Federal Information Processing Standard). On January 4, 2015 Zhigang Wang, an Oracle Solution Specialist reported the FIPS incompatibility and a submitted the patch changing the $512 bit$ prime to the $1024 bit$ number. This commit message was made by Gerhard Rieger the maintainer and lead developer but clearly sited Zhigang Wang for the patch.
\section{Why is this bad?}
\indent The first $p$ that was actually prime was pretty small, did not adhere to the FIPS standard and was much below the recommend minimum Diffie Hellman prime size.

``3 According to NIST SP 800‐131A, Diffie‐Hellman with 1024‐bit key sizes are deprecated through the end of 2013 and 2048‐bit key sizes will become the minimum. Users should start transitioning to the larger key sizes.''~\cite{FIPS}. 

This gives some credit back to Zhigang Wang for responsibly telling the Socat's developer this information, but if Zhigang Wang was the one to actually submit the non-prime $p$, what could he have actually done with it?
\\
\indent Lets first look at what the $p$ has to do with everything. Size does matter with the public prime numbers, as small primes make it easier for adversaries to attack the key agreement. The first $p$ was $512 bits$ which is not considered secure anymore, mainly because there are machines that could attack this size prime given it's computational power. If Zhigang Wang wanted to hack Socat it would have been easier to leave the prime as $512bits$ and not submit the patch for the larger prime $p$. Lets us assume that this
number was submitted for the purpose of creating a backdoor. How much work would that take?
\\
\indent You might wonder why having a non-prime $p$ could hinder the security of the DHKA, $p$ is public, everyone already knows it. The adversary is trying to solve for $g^{ab}$ given $g^a$ and $g^b$. This equation is known as the Diffie Hellman Problem or more generically a discrete logarithm problem (DLP). It is true that $p$ is public, but calculating discrete logs takes a lot of time and computational power.  One math trick that can speed
up modulus calculations is the Chinese Remainder Theorem (CRT). The CRT states that given any mathematical computation within $\mathbb{Z}_{p}$ the same calculation can be calculated $mod p$'s factors. With our modern computers we are able to perform calculations much easier on the smaller $mod$ factors of $p$.
\\
\indent The CRT is a very powerful function in cryptoanalysis. When a number can be broken into smaller prime factored parts, the number is considered smooth. The discrete logarithm problem can be solved in smaller groups and then using the CRT, you can obtain the discrete log of $g^{ab}$. The best current algorithm that will solve the discrete logarithm problem is Number Field Sieve, which holds the current record for a “random” modulo $p$ that is 530 bits long. So does a non-prime number help
in bruteforcing the DLP? It turns out no, not really.
\subsection{Backdoor Possibilities}
\noindent One way an adversary could attack the DHKA would be with Pohlig Hellman attack.\\
\indent To fully execute a Pohlig-Hellman attack we need to find $\Phi(p)$ also known as the Euler Totient of the modulus. Then, even if we separate the discrete logarithm problem into smaller subsets using the CRT, the size of the factors will still affect the difficulty of the DLP of the multiplicative group $\mathbb{Z}_{p}$. So far given the insecure prime $p$ used in Socat the factors that have been found so far are $a =$ 271, $b =$ 13587, and 
\\$c =$
388948843976343660073564545483323706469727242688027819734402088955429361655646564 \\
7352454140331039340582059836626167317380213077123632531487837183036372378804582171198 \\
546144167567931605824660910435516113447004 670533759317049846261619565037897529811714 \\
1144096886684800236261920005248055422089305813639519 \\ 
$c$ is a $1002 bit$ non-prime number that we are able to further factorize, but as for now the factors remain unknown. With trial division we are able to determine the smallest factors are larger than $2^{40}$. Trial division is a test where it will systematically test whether $p$ is divisible by any smaller number, very slowly as it checks every number in a Sieve of Eratosthenes manner. If we want to get anywhere near the factors of $c$ we will have to do something more efficient
like the Lenstra Elliptic Curve factorization. Lenstra algorithm's computation performance is based on the size of the largest factor not the entire number itself.  However, since there are still no solutions the factors of $c$ might be a product of two large $500 bit$ primes, which would still be very difficult to solve and may never be found. With no solutions for the factors of $c$ we are not able to take full advantage of the CRT in computing the DLP. 
\\
\indent The computations that would need to take place after the smaller factors of $c$ are found are with the Chinese remainder theorem. If the subsets are small enough we would be able to brute force $x_{1}$ and $x_{2}$ but if they are still too large we could use more efficient DLP algorithms. DLP can be solved more efficiently with either Pollar's Rho or index calculus in each of the smaller sub groups. For example:\\
\indent $\Phi(p)=q_{1}*q_{2}$\\
\indent find $x_{1}$ such that $h^{q_{1}}=(g^{q_{1}})^{x_1}(mod p)$ \\
\indent find $x_{2}$ such that $h^{q_{2}}=(g^{q_{2}})^{x_2}(mod p)$ \\
\indent Then we can prove that $x=x_{1}(mod q_{2})$ and $x=x_{2}(mod q_{1})$, and thus can figure out $x$ with the CRT. \\
\indent If an attacker could find $x_{1}$ and $x_{2}$ then they would be able generate the same key as the secured participants and would be able to decrypt any message passed between them, in this case it would have been the symmetrical key. This would give the adversary full access to any data sniffed over the wire~\cite{thai}.
\\
\indent As for now there have not been any known successful attacks on Socat due to the p not being prime. It is also clear to see that even though the p was not prime it was still very hard for any adversary to solve the DLP to launch an attack. However, it is still unclear how that number was put into place and could have been precalculated with certain factors that only Zhigang Wang knew.
\section{How the problem was fixed}
This problem was discovered by Santiago Zanella-Beguelin who is a part of Microsoft Vulnerability Research (MSVR) group. “Microsoft will never reveal vulnerability details before a vendor-supplied update is available for issues reported though the MSVR program unless there is significant evidence of active attacks in the wild. [MSVR]”. Microsoft Vulnerability Research discloses vulnerabilities to the third-parties with a Coordinated Vulnerability Disclosure (CVD)
approach~\cite{micro}.
\\
\indent The fix for Socat was simple. Generate a new Diffie Hellman $p$ parameter and update it to the new FIPS standard of $2048bits$. It is unclear if the non-prime value was actually ever exploited or put in place for a malicious backdoor. With the uncertainty of whether or not the non-prime value was a mistake, the analysis of security hard to determine. If the value was set in place as a backdoor then it very well could have been exploited, but if it was just a mistake it may not
have affected anything.
\\
Zhigang Wang was asked to comment on how the non-prime number that was submitted as a patch was created, to which no response was sent back. 
\section{Conclusion / What I learned}
\noindent To exploit the bug that was in the Socat DHKA you must do the following:
\begin{enumerate}
    \item Factor p to smaller values, most efficient would be Lenstra elliptic curve.
    \item Use Pohlig-Hellman and CRT to reduce DLP to the factor sets of $\mathbb{Z}_{p}$
    \item Use Pollar's rho or index calculus to solve DLP in each factor set of $\mathbb{Z}_{p}$
    \item Sniff Socat encrypted network traffic to recover public DH parameters
    \item Decrypt data
\end{enumerate}
\indent I also took note of the fact that you should always double check your security implementation, or have another person look it over to find stupid errors like a prime not being prime. It was also an eye opener as to how long it might be before someone finds a mistake in your code. The whole time your system is vulnerable you should assume that an adversary has compromised your application and consider all data that was used within that application compromised.  It was fun for me to look into
this topic. I liked how it felt like a mystery as to why the value was not prime. I came up with multiple conspiracy theories while writing this as to why the non-prime value was added. It still does not seem like a mistake but at this point I might just have to leave it at that, a mistake.

\medskip
 
\bibliography{crypto}
\bibliographystyle{ieeetr}
\end{document}
